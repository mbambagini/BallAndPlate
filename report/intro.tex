
%%% Local Variables:
%%% mode: latex
%%% TeX-master: "paper"
%%% End:

\section{Introduction}
\label{sec:intro}

The goal of this project is to develop a Ball and Plate balancing control
system.

The ball and plate system consists of a ball on a plate, capable of being
tilted along each of two horizontal axis of the plate itself.
The plate, initially horizontal, is tilted aiming to control the
position of the ball, so the ball is indirectly controlled.
%In this scenario, such a position is static.

We designed the balancing control system using a neural
network based on the reinforcement-learning model, that is 
the problem faced by an agent that learns behavior
through trial-and-error interactions with a dynamic environment.
Such an approach, not making use of a formalized system description 
as digital control techniques do, programs agents by reward and
punishment without needing to specify how such a \emph{task} is to be achieved.

The project consists in two programs, written in C, one for the learning phase,
(training the neural networks) and the other for the graphical representation
(using the neural networks) using the \emph{OpenGL} library for a 3D
simulation.

%The challenge of balancing is a problem under continuous study for
%applications from robotics to transportation, often extensions of the
%inverted pendulum project. Therefore, the system can present many challenges
%and opportunities as an educational tool to university students studying
%control systems engineering.

%This project adopts a simulated ball-on-plate system model.
%, implemented using the C language.
%In order to reproduce the ball-on-plate behaviour in a realistic way,
%a minimal but effective graphical layout
% using the Open Graphics Library (OpenGL)
%has been designed.
 
%Characterizing the ball-on-plate specs is difficult,
%notably because the desired motion is for an object of the system that is
%indirectly controlled.
%The controller can not directly manipulate the ball, and therefore deciding
%on the specs of the plate controller, in terms of rise time, settling time,
%and steady-state error is difficult.

%For these reasons, we designed the balancing control system using a neural
%network based on the reinforcement-learning model, that is 
%the problem faced by an agent that learns behavior
%through trial-and-error interactions with a dynamic environment.
%Such an approach, not making use of a formalized system description 
%as digital control techniques do, programs agents by reward and
%punishment without needing to specify how such a \emph{task} is to be achieved.

The rest of the report is organized as follows: Section~\ref{sec:sysmodel}
introduces a linearized ball and plate model, core of the implemented simulator.
Section~\ref{sec:k-ase} presents the system architecture together with the
neural network tuning.
%reinforcement-learning model used in this
%project together with the tuning strategies for the neural network parameters.
Section~\ref{sec:results} proves the effectiveness of our approach, showing
the results of the neural network training. Section~\ref{sec:impl}
contains a description of the developed programs and how to use them. 
Section~\ref{sec:conclusions} ends the report with the concluding remarks.

%As shown, the project needs to be broken into three subprojects: the sensor
%system, the physical system, and the controller. These three projects will be
%developed independently by the team, and then integrated into a complete
%system. The sensor subsystem consists of the circuitry and software that
%generate coordinates from the touch screen. The physical system refers to the
%design and machining of parts needed to construct the large yoke and axle, as
%well as the new motor bracket, which are necessary to bear the weight of the
%touch screen. The control system includes not only the modelling and simulation
%of the system as a whole, but additionally, the design of the controllers: one
%for the ball position and another for the plate angles.
%One of the major obstacles to integration and implementation is the reading of
%position coordinates from the touch. A RS-232 serial card is supplied with the
%touch screen, however, the average position report rate is 80-100
%positions/second. Another obstacle is the system itself, which is open-loop
%unstable: the design must take into account not only the dynamics of the two
%axis plate system, but also the dynamics of the heavy steel ball moving around
%on the surface of the system.

